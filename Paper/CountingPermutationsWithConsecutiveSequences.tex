
%          -------------------------------------------------------------------------------------------------------------
%          -- The Fixed Rate Equivalent (FREQ): Resolving the Multiple Solutions Issue of the Internal Rate of Return --
%          -------------------------------------------------------------------------------------------------------------
%          ---------------------------------------------------- by -----------------------------------------------------
%          -------------------------------------------------------------------------------------------------------------
%          ----------------------------------------------- Thomas Becker -----------------------------------------------
%          -------------------------------------------------------------------------------------------------------------
%          ------------------------------------------------- APR 2017 --------------------------------------------------

\documentclass{article}
\usepackage{amsfonts}
\usepackage{amstext}
\usepackage{amsmath}
\usepackage{amsfonts}
\usepackage{multirow}
\usepackage{hyperref}
\usepackage{xcolor}
\hypersetup{
    colorlinks,
    linkcolor={red!50!black},
    citecolor={blue!50!black},
    urlcolor={blue!80!black}
}
\usepackage{theorem}
\usepackage{enumerate}

% ---------------------------------------------------------------------
%
% Amstex math italic bold
%
% ---------------------------------------------------------------------
%
\font\tencmmib=cmmib10
\font\ninecmmib=cmmib9
\font\sevencmmib=cmmib7
\font\fivecmmib=cmmib5
%
\newfam\cmmibfam
\def\mib{\fam\cmmibfam\tencmmib}
%
% NOTE: If these fonts are not available, use the following alternate
% definition of \mib:
%\def\mib{\mathbf}
%
\textfont\cmmibfam=\tencmmib
\scriptfont\cmmibfam=\sevencmmib
\scriptscriptfont\cmmibfam=\fivecmmib

{\theoremstyle{plain}
  \theorembodyfont{\rm}
  \newtheorem{theorem}{Theorem}[section]
  \newtheorem{proposition}[theorem]{Proposition}
  \newtheorem{lemma}[theorem]{Lemma}
  \newtheorem{corollary}[theorem]{Corollary}
  \newtheorem{definition}[theorem]{Definition}
  \newtheorem{example}[theorem]{Example}
  \newtheorem{notation}[theorem]{Notation}
  \newtheorem{purpose}{Purpose of this Section}
}
\renewcommand{\thepurpose}{\kern -5pt}
\renewcommand\refname{References and Notes}

\def\proof{\par \noindent              %--- PROOF ----------------------
\mbox{\bf Proof}\ \,}                  %
\def\endproof{\mbox{$\Box$} \par }     %--- END OF PROOF ---------------

% Use vertical space instead of indent to start a paragraph
%
\setlength{\parindent}{0pt}
\setlength{\parskip}{1ex plus 0.5ex minus 0.2ex}

\begin{document}
\author{Thomas Becker\\ {\small \it thomas@greaterthanzero.com}}
\title{Two General Purpose Algorithms for Counting Permutations with Consecutive Sequences}
\date{\small April 22, 2017}
\maketitle

\begin{abstract}
  We state, prove the correctness of, and discuss the complexity of two general
  purpose algorithms, one building upon the other, for counting permutations with
  specified configurations of consecutive sequences. The algorithms are based on
  a theorem that describes how counting permutations with consecutive sequences can
  be reduced to counting permutations with no consecutive sequences at all.
\end{abstract}

\section{Preface}
Combinatorial mathematics is not my specialty as a mathematician. However, I recently wrote
a rather lightweight blog post \cite{BlogPost} on the subject of random shuffle mode on today's
music players. In the process, I needed to know the number of permutations with certain
configurations of consecutive sequences. The answers to many of my questions were readily
available on the Web (see e.g. \cite{Oeis}),
but there were also some questions to which I did not find the answer. Therefore, I sat down
and wrote some general-purpose algorithms---based on one core mathematical theorem---to conveniently
calculate the numbers that I needed. The algorithms are available on github \cite{Algos}.
I checked the results against the Online Encyclopedia of Integer Sequences \cite{Oeis} and
against brute force for small numbers of elements. But since the correctness of the algorithms
is far from obvious, I also felt that formal mathematical correctness proofs should be available.
That is the reason why I wrote this paper. I am quite certain that there is nothing here that
combinatorial mathematicians don't already know, but again, I was not able to quite find the things I
needed online. Any feedback, particularly regarding references to
the literature, would be much appreciated.

\section{Introduction}

\begin{notation}
  If $n$ is an integer with $n \geq 1$, we write ${\mathcal P}_n$ for the set of all permutations of the integers
  $1, 2, \ldots, n$.
\end{notation}

\begin{definition}
  Let $P\in {\mathcal P}_n$, and let $k \geq 2$.
  A {\bf consecutive sequence} of length $k$ in $P$ is a contiguous subsequence of $k$ consecutive
  integers in $P$, that is, a contiguous subsequence of the form $(i, i+1, \ldots, i+k-1)$.
  A consecutive sequence is called {\bf maximal} if it is not a subsequence of a consecutive sequence
  of greater length.
\end{definition}

There is a vast body of work regarding the count of permutations that have a specified configuration of
consecutive sequences, such as permutations having a certain number of consecutive pairs or triples
\cite{OeisRefs}. In this paper, we state, prove the correctness of, and discuss the complexity of two general
purpose algorithms, one based upon the other, for
counting permutations whose maximal consecutive sequences are described in certain ways. The need
for these algorithms, which are available under the MIT license \cite{Algos}, arose from the author's curiosity
about the behavior of random shuffle mode on today's music players \cite{BlogPost}. The algorithms do not
use the (more generally applicable) inclusion-exclusion principle that is often employed for counting
permutations with certain properties. Instead, they rely on a technique of reducing the problem of counting
permutations with consecutive sequences to the problem of counting permutations with no consecutive sequences
at all. This technique is a generalized form of an argument that the author encountered in a Quora post by Jed Yang
\cite{JedYang}.

Our core algorithms deal with the number of permutations having certain configurations of
     {\it maximal} consecutive sequences. However, as we will show, they can be employed to answer questions
     regarding just consecutive sequences as well.

The organization of the paper is as follows. In Theorem \ref{theorem_count_by_length_and_initial_element},
we give an auxiliary algorithm on which the
two main algorithms are based. It calculates the number of permutations that meet a specification of
maximal consecutive sequences by initial elements and lengths.

Theorem \ref{theorem_count_by_length_and_count}
provides an algorithm to count the permutations whose maximal consecutive sequences have been specified by
stating how many exactly of each length there should be. Using that algorithm, one may, for example,
calculate the number of permutations that have exactly three consecutive pairs, none of which are linked
to form a consecutive triple (specify ``three maximal consecutive sequences of length two, zero maximal
consecutive sequences of any other length'').

Building on top of that, Theorem \ref{theorem_iterating_over_specs_by_count_by_length_and_count}
describes a generic, customizable algorithm
that iterates over configurations of maximal consecutive sequences by lengths and counts.
A user-supplied function decides if permutations with a given configuration should be included in the count or not.
For example, the user-supplied function could accept any configuration that specifies a non-zero count for maximal
sequences of length two and a zero count for all other lengths, and reject all others configurations.
The result would be the number
of permutations that have any number of consecutive pairs, but no linked pairs that form larger
consecutive sequences. When the algorithm iterates over {\it all} possible specifications of maximal
consecutive sequences by lengths and counts, its complexity, although dramatically better than a brute force
solution to the problem, leaves a lot to be desired. However, our implementation provides custumizability
that can lead to vastly improved complexity in many cases, such as the example that we just mentioned.

Finally, the last section of this paper discusses how to use our core algorithms, which deal with
maximal consecutive sequences, to answer questions regarding just plain consecutive sequences.

\section{Specifying Maximal Consecutive Sequences by Lengths And Initial Elements}

\begin{definition} \label{def_mcs_spec_by_length_and_initial_element}
  Let $n$ be an integer with $n\geq 1$. An {\bf MCS-specification by lengths and initial elements} for $n$ is a
  set of pairs of integers $$\{\,(a_1, k_1), (a_2, k_2), \ldots,(a_m, k_m)\,\}$$ with the following properties:
  \begin{enumerate}[(i)]
  \item
    $a_i \geq 1$ and $k_i \geq 2$ for $1\leq i \leq m$,
  \item
    $a_i + k_i \leq a_{i+1}$ for $1\leq i \leq m-1$,
  \item
    $a_m + (k_m-1) \leq n$.
  \end{enumerate}
\end{definition}

\begin{notation}\label{notation_mcs_spec_by_length_and_initial_element}
  If S is an MCS-specification by lenghts and initial elements for $n$ as in
  Definition \ref{def_mcs_spec_by_length_and_initial_element}
  above, we write ${\mathcal Q}_{(n,S)}$ for the set of all permutations
  $P \in {\mathcal P}_n$ with the following property: for each $i$ with $1\leq i \leq m$, $P$ has a maximal
  consecutive sequence of length $k_i$ that starts with the integer $a_i$, and $P$ has no other maximal
  consecutive sequences.
\end{notation}

\begin{purpose}
Present an auxiliary algorithm, to be used in later sections,
for calculating $|{\mathcal Q}_{(n,S)}|$ from $n$ and $S$.
\end{purpose}

The following technical lemma will be needed when we use induction on $m$ in connection with
MCS-specifications by lengths and initial elements.

\begin{lemma}\label{lemma_ind_on_m} 
  Let $S$ be an MCS-specification by lengths and initial elements for $n$ as in
  Definition \ref{def_mcs_spec_by_length_and_initial_element}
  above, and assume that $m\geq 1$. Then $n-(k_m-1) \geq 1$, and
  $$T = \{\,(a_1, k_1), (a_2, k_2), \ldots,(a_{m-1}, k_{m-1})\,\}$$
  is an MCS-specification by lengths and initial elements for $n-(k_m-1)$.
\end{lemma}

\proof
From (i) and (iii) of Definition \ref{def_mcs_spec_by_length_and_initial_element}, we may conclude that
$$1 \leq a_m \leq n - (k_m - 1),$$ which proves the first claim of the lemma.
If $m=1$, the second claim is trivial since the empty set is an MCS-specifications by lengths
and initial elements for any positive integer. Now let $m > 1$. It is clear that $T$ has properties
(i) and (ii) of Definition \ref{def_mcs_spec_by_length_and_initial_element}. Moreover, we have 
\begin{eqnarray*}
  a_{m-1} + (k_{m-1} - 1) & \leq & a_m - 1 \\
                          & \leq & n - k_m \\
                          & \leq & n - (k_m-1),
\end{eqnarray*} 
and thus $T$ satisfies (iii) of Definition \ref{def_mcs_spec_by_length_and_initial_element} as well.
\endproof

\begin{notation}
  We let ${\mathcal U}_n$ denote the subset of ${\mathcal P}_n$ that consists of all permutations with no consecutive
  sequences.
\end{notation}

It is well-known (see e.g. \cite{JedYang}) that the cardinality of ${\mathcal U}_n$ satisfies the recurrence relation
$$
|{\mathcal U}_n| = (n-1) \cdot |{\mathcal U}_{n-1}| + (n-2) \cdot |{\mathcal U}_{n-2}|.
$$
The theorem below provides the desired algorithm for calculating $|{\mathcal Q}_{(n,S)}|$ by reducing the
problem to the calculation of $|{\mathcal U}_r|$ for a certain $r$.
 
\begin{theorem}\label{theorem_count_by_length_and_initial_element}
  Let $n \geq 1$, let $S = \{\,(a_1, k_1), (a_2, k_2), \ldots,(a_m, k_m)\,\}$ be an
  MCS-specification by lengths and initial elements for $n$, and let $k = \sum_{i=1}^m k_i$.
  Then $|{\mathcal Q}_{(n,S)}| = |{\mathcal U}_{n - (k - m)}|$.
\end{theorem}

\proof
We will prove the theorem by showing that there is a bijection between ${\mathcal Q}_{(n,S)}$
and ${\mathcal U}_{n - (k - m)}$. For this, it suffices to show that there are maps
$$
f: {\mathcal Q}_{(n,S)}\rightarrow {\mathcal U}_{n - (k - m)}
\quad\text{and}\quad 
g: {\mathcal U}_{n - (k - m)} \rightarrow {\mathcal Q}_{(n,S)}
$$
such that $g\circ f$ is the identity on ${\mathcal Q}_{(n,S)}$ and $f\circ g$ is the identity on
${\mathcal U}_{n - (k - m)}$. Intuitively speaking, $f$ is obtained by throwing out
all elements of maximal consecutive sequences except for the initial ones, then adjusting greater
elements of the permutation downward to close the gaps. The map $g$ is the reverse operation of that.
For a formal proof of the existence of these maps, we proceed by induction on $m$. For $m=0$,
the claim is trivial as $${\mathcal U}_{n - (k - m)} = {\mathcal U}_n = {\mathcal Q}_{(n,S)}$$
in that case. Now let $m>0$, and let
$$
T = \{\,(a_1, k_1), (a_2, k_2), \ldots,(a_{m-1}, k_{m-1})\,\}.
$$
By Lemma \ref{lemma_ind_on_m}, $T$ is an MCS-specification by lengths and initial elements
for $n - (k_m - 1)$. This together with the induction hypothesis implies
that there is a bijection between
$$
{\mathcal Q}_{(n-(k_m-1),T)}
\quad\text{and}\quad
{\mathcal U}_{(n-(k_m-1)) - ((k-k_m) - (m-1))} = {\mathcal U}_{n - (k - m)}.
$$
Therefore, it suffices to construct maps
$$
f: {\mathcal Q}_{(n,S)}\rightarrow {\mathcal Q}_{(n-(k_m-1),T)}
\quad\text{and}\quad 
g: {\mathcal Q}_{(n-(k_m-1),T)} \rightarrow {\mathcal Q}_{(n,S)}
$$
such that $g\circ f$ is the identity on ${\mathcal Q}_{(n,S)}$ and $f\circ g$ is the identity on
${\mathcal Q}_{(n-(k_m-1),T)}$.
For $P\in {\mathcal Q}_{(n,S)}$, let $f(P)$ be the integer sequence that is obtained from $P$ as follows:
\begin{enumerate}
\item
  Strike the elements $a_m + 1, a_m+2, \ldots, a_m+(k_m-1)$ from $P$.
\item
  In the remaining sequence, replace every element $a$ that is greater than $a_m$ with $a-(k_m-1)$.
\end{enumerate}
For $Q\in {\mathcal Q}_{(n-(k_m-1),T)}$, first note that the integer $a_m$ occurs in the sequence $Q$
because $a_m \leq n-(k_m-1)$ by Definition \ref{def_mcs_spec_by_length_and_initial_element} (iii).
Now let $g(Q)$ be the integer sequence that
is obtained from $Q$ by reversing the procedure that defines $f$:
\begin{enumerate}
\item
 Replace every element $a$ in $Q$ that is greater than $a_m$ with $a+(k_m-1)$.
\item
  Augment the resulting sequence by inserting the sequence $(a_m + 1, a_m+2, \ldots, a_m+(k_m-1))$
  following the element $a_m$.
\end{enumerate}
It is easy to see that $f(P)$ contains exactly the integers $1, 2, \ldots, n-(k_m-1)$, and $g(Q)$ contains
exactly the integers $1, 2, \ldots, n$, and therefore,
$$
f(P)\in {\mathcal P}_{n-(k_m-1)}
\quad\text{and}\quad
g(Q)\in {\mathcal P}_n.
$$
Also, it is immediate from the definition of $f$ and $g$ that $g\circ f$ is the
identity on ${\mathcal Q}_{(n,S)}$ and $f\circ g$ is the identity on ${\mathcal Q}_{(n-(k_m-1),T)}$.
It remains to show that
$$
f({\mathcal Q}_{(n,S)}) \subseteq {\mathcal Q}_{(n-(k_m-1),T)}
\quad \text{and}\quad
g({\mathcal Q}_{(n-(k_m-1),T)}) \subseteq {\mathcal Q}_{(n,S)}.
$$
So let $P\in {\mathcal Q}_{(n,S)}$. To show that $f(P)\in {\mathcal Q}_{(n-(k_m-1),T)}$, we must prove
that $f(P)$ has precisely the maximal consecutive sequences that $T$ specifies. Before delving into that argument,
it may be helpful to visualize how $f(P)$ is obtained from $P$. Under the action of $f$, an element of the
sequence $P$ may be removed, change its position, change its value, change both position and value,
or change neither position nor value. The elements $a_m + 1, a_m + 2, \ldots, a_m + (k -1)$, which we know
are positioned consecutively, get removed. The elements that are positioned to the right of that subsequence,
all the way to the end of $P$, move $k_m - 1$ positions to the left. Finally, those elements are greater than
$a_m$---and the only ones that are left are actually greater than $a_m + (k -1)$---are decremented by $k-1$.
You may also want to remind yourself that the subscript $m$ on $a_m$ is not indicative of position in
$P$ or $f(P)$. It stems from the MCS-specification $S$.

Now imagine the sequence $f(P)$ being split in two, with the cut being after the element
$a_m$. Let's call these two pieces $P_1$ and $P_2$. All the integers that are members of the $m-1$ maximal
consecutive sequences in $P$ starting with $a_1, a_2, \ldots, a_{m-1}$ are less than $a_m$. Therefore,
their values are not changed under the action of $f$, and neither are their relative positions.
Therefore, each of these sequences is present as a consecutive sequence in either $P_1$ or $P_2$. 
As for the elements in between and around those sequences, in $P_1$ or $P_2$,
they are either less than or equal to $a_m$,
in which case their value is unchanged under $f$, or they are greater than $a_m$, in which case they are
the result of decrementing in lockstep, by the same amount, namely, $k_m -1$. Moreover, no relative positions
have changed among any of these under the action of $f$. It follows that none of these elements
have joined any of the maximal consecutive sequences of $P$, and the only new consecutive pair that
could have formed among them would be $(a_m, a_m+1)$, but that's impossible since $a_m$ sits at the
end of $P_1$. We see that the maximal consecutive sequences that we find in $P_1$ and $P_2$ are precisely those
that are specified by $T$.

It remains to show that no consecutive pair forms between the last element of $P_1$ and the first element of
$P_2$ as we join the two to form $f(P)$. The last element of $P_1$ is $a_m$. The first element of $P_2$ is
the result of the effect that $f$ had on the first element following the maximal consecutive sequence
$a_m, a_m + 1, \ldots, a_m + (k -1)$ in $P$. That element was either less than $a_m$, in which
case its value is unchanged, or it was greater than $a_m + (k_m-1)$ and unequal to $a_m + k$, in which case its value
was changed to something not equal to $a_m + 1$. In either case, no consecutive pair forms at the juncture
of $P_1$ and $P_2$. This concludes the proof that $f(P)\in {\mathcal Q}_{(n-(k_m-1),T)}$ and thus
$f({\mathcal Q}_{(n,S)}) \subseteq {\mathcal Q}_{(n-(k_m-1),T)}$. We leave the proof of
$g({\mathcal Q}_{(n-(k_m-1),T)}) \subseteq {\mathcal Q}_{(n,S)}$ to the reader, as it is little more than
the argument that we just made in reverse.
\endproof

Since we know how to calculate the cardinality of ${\mathcal U}_n$ for any $n$, the theorem above gives
us an algorithm to calculate the number of permutations of $n$ integers that have maximal consecutive
sequences of specified lengths with specified initial elements. However, judging from experience, that algorithm
isn't very interesting. The description of consecutive sequences is just too specific. What one wants
is being able to count the permutations with consecutive sequences or maximal consecutive sequences that
are specified by lengths and counts, as in, ``exactly $x$ number of consecutive triples,'' or, ``exactly
$x$ number of consecutive triples and no longer consecutive sequences,'' or some such thing. This will
be achieved in the next two sections.

As for the complexity of the algorithm of Theorem \ref{theorem_count_by_length_and_initial_element},
it is clear that the classical recurrence relation for ${\mathcal U}_n$ that we stated preceding the
theorem can be rewritten as a bottom-up multiplication that calculates ${\mathcal U}_n$ in constant
space and linear time. Therefore, the complexity of the algorithm of
Theorem \ref{theorem_count_by_length_and_initial_element} is $O(n)$.

As an aside, let us mention that Theorem \ref{theorem_count_by_length_and_initial_element} continues
to hold if instead of specifying maximal consecutive sequences by initial element and count, we specify
them by initial position and count. This follows from the fact that for $n\geq 1$,
the map that exchances value and position is a bijection on ${\mathcal P}_n$. Here, the permutation
$(a_1, a_2, \ldots, a_m)$ maps to the permutation where $i$ is the element at position $a_i$ for $1 \leq i\leq n$. 
Under this map, maximal consecutive sequences of length $k$ with initial element $a$ map to maximal
consecutive sequences of length $k$ that start at position $a$ and vise versa.

\section{Specifying Maximal Consecutive Sequences by Lengths And Counts}

\begin{definition} \label{def_mcs_spec_by_length_and_count}
  Let $n$ be an integer with $n\geq 1$. An {\bf MCS-specification by lengths and counts} for $n$ is a
  set of pairs of integers $$\{\,(k_1, c_1), (k_2, c_2), \ldots,(k_m, c_m)\,\}$$ with the following properties:
  \begin{enumerate}[(i)]
  \item
    $k_i \geq 2$ and $c_i \geq 1$ for $1\leq i \leq m$,
  \item
    $\sum_{i=1}^m c_i \cdot k_i \leq n$, and
  \item
    $k_i \neq k_j$ for $1\leq i, j \leq m$.
  \end{enumerate}
\end{definition}

\begin{notation}\label{notation_mcs_spec_by_length_and_count}
  If $T$ is an MCS-specification by lenghts and counts for $n$ as in
  Definition \ref{def_mcs_spec_by_length_and_count}
  above, we write ${\mathcal R}_{(n,T)}$ for the set of all permutations
  $P \in {\mathcal P}_n$ with the following property: for each $i$ with $1\leq i \leq m$, $P$ has exactly $c_i$
  maximal consecutive sequence of length $k_i$, and $P$ has no other maximal
  consecutive sequences.
\end{notation}

\begin{purpose}
Present an algorithm for calculating $|{\mathcal R}_{(n,T)}|$
from $n$ and $T$.
\end{purpose}

It is clear from Definitions \ref{def_mcs_spec_by_length_and_initial_element}
and \ref{def_mcs_spec_by_length_and_count} and the corresponding Notations
\ref{notation_mcs_spec_by_length_and_initial_element} and \ref{notation_mcs_spec_by_length_and_count} that
${\mathcal R}_{(n,T)}$ is the disjoint union of certain ${\mathcal Q}_{(n,S)}$, namely, those
where $S$ ranges over all those
MCS-specifications by lengths and initial elements that are of the form
$$S = \{\,(a_1, l_1), (a_2, l_2), \ldots,(a_p, l_p)\,\}$$
with the properties
\begin{enumerate}[(i)]
\item
  $p = \sum_{i=1}^m c_i$, and 
\item
  for $1\leq i \leq m$, there are exactly $c_i$ many $j$ with $1\leq j \leq p$ and $l_j = k_i$.
\end{enumerate}

So if we denote the set of all MCS-specifications by lengths and initial elements that satisfy (i) and (ii)
above by ${\mathcal S}_T$, then we have, as a first step towards our algorithm for calculating
$|{\mathcal R}_{(n,T)}|$,
$$
|{\mathcal R}_{(n,T)}| = \sum_{S\in {\mathcal S}_T} |{\mathcal Q}_{(n,S)}|.\eqno(1)
$$

Theorem \ref{theorem_count_by_length_and_initial_element} tells us how to calculate $|{\mathcal Q}_{(n,S)}|$,
and moreover, the algorithm for doing so uses only $n$, $p$, and $\sum_{j=1}^p l_j$.
It is immediate from properties (i) and (ii) above that
$$
p = \sum_{i=1}^m c_i
\quad \text{and}\quad 
\sum_{j=1}^p l_j = \sum_{i=1}^m c_i \cdot k_i.
$$
So if we let $c = \sum_{i=1}^m c_i$ and $k = \sum_{i=1}^m c_i \cdot k_i$, we can extend equation (1) above
to the following second step towards our algorithm for calculating $|{\mathcal R}_{(n,T)}|$:
$$
|{\mathcal R}_{(n,T)}| = |{\mathcal S}_T| \cdot |{\mathcal U}_{n - (k - c)}|.\eqno(2)
$$

Therefore, all that remains to do is to figure out what $|{\mathcal S}_T|$ is:
how many MCS-specifications by lengths and initial elements
are there that satisfy (i) and (ii) above? That number is fairly easy to describe: it is the number of ways
in which one can choose subsets
$A_1, A_2, \ldots, A_{m}\subset \{\,1, 2, \ldots, n\,\}$ such that  

\begin{enumerate}[(a)]
\item
  $|A_i| = c_i$ for $1\leq i\leq m$, and
\item
  the elements of the $A_i$ are far enough apart so that each $a \in A_i$ can
  be the initial value of a maximal consecutive sequence of length $k_i$.
\end{enumerate}

At first glance, it may seem difficult to figure out the number of ways in which the $A_i$ can be chosen.
The key to making it easy lies in going back the proof of the equality
$|{\mathcal Q}_{(n,S)}| = |{\mathcal U}_{n - (k - c)}|$ of
Theorem \ref{theorem_count_by_length_and_initial_element}, which we just used to pass from equation (1) to
equation (2). This equality was proved
by exhibiting a bijection between ${\mathcal Q}_{(n,S)}$ and ${\mathcal U}_{n - (k - c)}$.
We mapped permutations with maximal consecutive sequences to shorter permutations without any consecutive
sequences by striking from all maximal consecutive sequences all elements except for the
first one, then renumbering the remaining elements to close the resulting gaps. The inverse operation consisted of
starting with a permutation with no consecutive sequences, then
blowing up the specified initial elements to consecutive sequences by inserting and renumbering elements. At the risk
of being accused of a hand-waving argument, we'll say that it is now clear that picking the subsets
$A_1, A_2, \ldots, A_{m}\subset \{\,1, 2, \ldots, n\,\}$ with properties (a) and (b) above is equivalent to picking
subsets $B_1, B_2, \ldots, B_{m}\subset \{\,1, 2, \ldots, n - (k - c)\,\}$ with just property (a).
The formal proof by induction parallels the proof of Theorem \ref{theorem_count_by_length_and_initial_element} and
is simpler than the latter.
Counting the ways in which the $B_i$ can be selected is elementary. The answer is
$$
\prod_{i=1}^m\binom{n-(k-c) - \sum_{j=1}^{i-1}c_j}{c_i}\,,
$$
or, equivalently,
$$
\frac{(n-(k-c))!}{c_1! \cdot c_2! \cdot \ldots \cdot c_m! \cdot c!}\,,
$$
or, equivalently,
$$
\frac{(n-(k-c)) \cdot (n-(k-c) - 1)\cdot \ldots \cdot (n-(k-c) - c + 1)}{c_1! \cdot c_2! \cdot \ldots \cdot c_m!}\,.
$$

We have thus proved the following theorem, which provides the desired algorithm for calculating
$|{\mathcal R}_{(n,T)}|$ from $n$ and $T$.

\begin{theorem}\label{theorem_count_by_length_and_count}
  Let $n \geq 1$, let $T = \{\,(k_1, c_1), (k_2, c_2), \ldots,(k_m, c_m)\,\}$ be an
  MCS-specification by lengths and counts for $n$, let $k = \sum_{i=1}^m c_i \cdot k_i$,
  and let $c = \sum_{i=1}^m c_i$.
  Then
  $$
  |{\mathcal R}_{(n,T)}| = |{\mathcal U}_{n - (k - c)}| \cdot
  \prod_{i=1}^m\binom{n-(k-c) - \sum_{j=1}^{i-1}c_j}{c_i},
  $$
  or, equivalently,
  $$
  |{\mathcal R}_{(n,T)}| = |{\mathcal U}_{n - (k - c)}| \cdot
  \frac{(n-(k-c))!}{c_1! \cdot c_2! \cdot \ldots \cdot c_m! \cdot c!}\,,
  $$
  or, equivalently,
  \begin{eqnarray*}
    |{\mathcal R}_{(n,T)}| & = & |{\mathcal U}_{n - (k - c)}| \cdot \\
                           &   &
    \frac{(n-(k-c)) \cdot (n-(k-c) - 1)\cdot \ldots \cdot (n-(k-c) - c + 1)}{c_1! \cdot c_2! \cdot \ldots \cdot c_m!}\,.
    \quad \Box
  \end{eqnarray*}
 
\end{theorem}

It is clear that the complexity of the algorithm of \ref{theorem_count_by_length_and_count}
is $O(m\cdot n)$, which, depending on how the $k_i$ are defined, can be anything between
$O(n)$ and $O(n^2)$.

\section{Iterating over Specifications by Lengths And Counts}

\begin{purpose}
  Present a generic algorithm for counting the permutations that meet certain specifications
  by lengths and counts, where a client-supplied function performs the selection of
  specifications to be included in the count.
\end{purpose}

Now that we know how to calculate $|{\mathcal R}_{(n,T)}|$, that is, the number of permutations that
meet a given specification by lengths and counts, it is an obvious and rather trivial thing to write
an algorithm that performs an in-place creation of every specification by lengths and counts for a
given $n$ and lets a user-provided function decide which ones should be included in the count.
Therefore, the following theorem requires no further proof.

\begin{theorem}\label{theorem_iterating_over_specs_by_count_by_length_and_count}
  Let $n \geq 1$, let ${\mathcal T}$ be the set of all MCS-specifications by
  lengths and counts for $n$, and let $f$ be a function from ${\mathcal T}$ to the set $\{\, 0, 1\,\}$.
  Then the expression
  $$
  \sum_{\{\,T \in{\mathcal T}\,\mid\, f(T) = 1\,\}}|{\mathcal R}_{(n,T)}|\eqno(3)
  $$
  amounts to an algorithm for calculating the number of permutations that meet exactly
  those MCS-specifications by lengths and counts for $n$ on which the function $f$ returns $1$.
  \endproof
\end{theorem}

The problem with this algorithm is that the number of MCS-specifications by lengths and counts
for $n$ grows faster with $n$ than one would wish. By definition, the number of these specifications is
\begin{eqnarray*}
  \big|\{\,(k_1, c_1), (k_2, c_2), \ldots,(k_m, c_m) & \mid & k_i \geq 2,\ c_i \geq 1 \text{ for } 1\leq i \leq m,\\
                                                 &      & \sum_{i=1}^m c_i \cdot k_i \leq n,\
                                                          k_i\neq k_j\text{ for } 1\leq i,j \leq m \,\}\big|\,.
\end{eqnarray*}
Determining what this is looks like a non-trivial combinatorial problem unto itself.
At this point, the best we know how to do is to look at some numbers. The brute force
approach to counting permutations begins to encounter performance issues at $n=12$, as
$12!=4.79001600\times 10^8$, and performance degrades quickly after that.
A comparable number of MCS-specifications by lengths and counts,
namely, $4.83502844\times 10^8$, is reached for $n=108$. Indeed, the algorithm of Theorem
\ref{theorem_iterating_over_specs_by_count_by_length_and_count} starts to noticeably slow
down for $n$ around $100$. Considering that $60!$ is roughly equal to the number of atoms
in the known, observable universe, being able to count permutations for $n=100$ must be
considered an achievement. On the other hand, $100$ is not exactly, shall we say, a large number.

Luckily, many common questions regarding the number of permutations with certain consecutive
sequences allow an optimization that can cut down dramatially on the number of MCS-specifications
by lengths and counts that need to be considered in the sum in (3) above. Typically, when counting
permutations with certain configurations of consecutive pairs, triples, quadruples, etc.
(maximal or not necessarily maximal), one knows in advance that the selection function $f$ of
Theorem \ref{theorem_iterating_over_specs_by_count_by_length_and_count} will reject every
MCS-specification by lengths and counts that specifies a non-zero count for maximal consecutive
sequences of length greater than some bound and/or less than some bound. To exploit that,
our implementation of the algorithm \cite{Algos} lets the user specify a
lower and/or upper bound for non-zero lengths
of maximal consecutive sequences. If $l$ and $u$ are the specified bounds, then the algorithm
will include only those MCS-specifications by lengths and counts in the sum in (3) above that
specify 0 for any length less than $l$ and greater than $u$.

For example, when
calculating the number of permutations that have maximal consecutive pairs but no other maximal
consecutive sequences, that is, permutations that have any number of consecutive pairs none of
which are linked to form longer consecutive sequences, one would tell the algorithm
to only generate those MCS-specifications by lengths and counts that specify a zero count for all lengths
greater than 2. This cuts the length of the sum in (3) above down to $\lfloor\frac{n}{2}\rfloor + 1$. 

Sometimes, it takes a bit of creativity to avail oneself of the lower bound/upper bound optimization. Suppose,
for example, that you wish to calculate the number of permutations that have at least one maximal consecutive
sequence of length greater than or equal to $k$ for some $k$.
As it stands, this condition does not allow you to use the lower bound/upper bound optimization.
But you could also calculate the number of permutations that have no maximal consecutive sequences
greater than or equal to $k$ and then subtract that from $n!$. Now the optimization is applicable.

\section{Permutations with Consecutive Sequences}
The algorithms we have discussed so far deal with the number of permutations having certain configurations
of maximal consecutive sequences. Oftentimes, one is interested in the number of permutations having certain
kinds of---not necessarily maximal---consecutive sequences. Adapting our algorithms for that purpose is
rather straightforward, and in some cases trivial. For a trivial case, consider the question, ``How many
permutations are there in ${\mathcal P}_n$ that have consecutive sequences of length $k$?'' Having a
consecutive sequence of length $k$ is obviously equivalent to having a maximal consecutive sequence of
length greater than or equal to $k$. Therefore, this is the application of Theorem
\ref{theorem_iterating_over_specs_by_count_by_length_and_count} that we mentioned at the end of the
previous section.

Perhaps the most commonly asked question about consecutive sequences is, ``How many
permutations are there in ${\mathcal P}_n$ that have $c$ many consecutive sequences of length $k$?''
To use our algorithms for answering this question, we need the following lemma whose proof is trivial.

\begin{lemma}
  Let $n \geq 1$, let $T = \{\,(k_1, c_1), (k_2, c_2), \ldots,(k_m, c_m)\,\}$ be an
  MCS-specification by lengths and counts for $n$. Furthermore, let $P\in {\mathcal R}_{(n,T)}$,
  that is, $P$ is a permutation that meets the specification $T$, and let $k \geq 2$.
  Then the number of consecutive
  sequences of length $k$ in $P$ equals
  $$
  \sum_{\shortstack{$\scriptstyle i=1$\\$\scriptstyle k_i\geq k$}}^{m}c_i\cdot (k_i - k + 1)\,.
  $$
\end{lemma}

It is now a straightforward task to count the permutations that have exactly $c$ consecutive sequences
of length $k$: apply Theorem \ref{theorem_iterating_over_specs_by_count_by_length_and_count} with a
selection function that employs the lemma above to accept exactly those MCS-specifications by lengths
and counts that result in $c$ consecutive sequences. Note also that the upper bound optimization that
we mentioned following Theorem \ref{theorem_iterating_over_specs_by_count_by_length_and_count} is
applicable here. That's because we know that any MCS-specification by lengths and count that specifies
a non-zero count for a length $l$ with $l > k + c - 1$ will be rejected. Therefore, we only need to
look at MCS-specifications by lengths and counts that specify a zero count for lengths greater than
$k + c - 1$. Our algorithm package on github \cite{Algos} has a ready-to-use implementation.

\bibliographystyle{alpha}
\begin{thebibliography}{0}

\bibitem{Oeis}
  \href{http://oeis.org/}{Online Encyclopedia of Integer Sequences}

\bibitem{OeisRefs}


  Even if the author's mathematical specialty were combinatorics, which it is not,
  it would be foolish to attempt an overview or an even remotely complete set of references
  in a short article like this. A good place to learn about existing results and start finding
  references is the \href{http://oeis.org/}{Online Encyclopedia of Integer Sequences}, specifically
 the entries \href{http://oeis.org/A010027}{A010027}, \href{http://oeis.org/A002628}{A002628}, and
  \href{http://oeis.org/A000255}{A0000255}.

\bibitem{Algos}
    \href{https://github.com/walkswiththebear/ConsecutiveSequences}{ConsecutiveSequences} on github.

\bibitem{BlogPost}
    \href{http://blog.greaterthanzero.com/post/159874910652/some-mathematics-algorithms-and-probabilities}{Some Mathematics, Algorithms, and Probabilities Concerning Your Music Player’s Random Shuffle Mode} at the GreaterThanZero company blog. 

\bibitem{JedYang}
 This \href{https://www.quora.com/What-is-the-probability-that-a-shuffled-music-album-will-have-at-least-two-songs-in-their-original-relative-consecutive-order}{proof by Jed Yang}
  on Quora is short, elegant, and self-contained.
\end{thebibliography}

\end{document}
